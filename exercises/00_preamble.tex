\centering{Die \textbf{Abgabefrist} dieser Bonusübung ist am \textbf{15.07.2020} um \textbf{23:59 Uhr}.}

\centering{Die Bonusübung wird am \textbf{16.07.2020} um \textbf{13:30 Uhr} besprochen.}

\begin{table}[h!]
\centering
\begin{tabular}{c|c|c|c|c|c|c|c|c}
\toprule
\textbf{Aufgabe}              & 1  & 2 & 3 & 4 & 5  & 6  & 7 & 8 \\ \hline
\textbf{Maximal Punktzahl}    & 16 & 4 & 3 & 4 & 15 & 19 & 6 & 7  \\ \hline
\textbf{Erreichte Punktzahl}  &   &   &   &   &   &   &   &    \\
\bottomrule
\end{tabular}
\end{table}

\begin{table}[h!]
\centering
\begin{tabular}{c|c|c|c}
\toprule
\textbf{Gruppe \groupID}              & \textbf{Nachnahme} & \textbf{Vorname} & \textbf{Matrikelnummer} \\
\midrule
\textbf{1}    & \FirstGroupMemberLastName & \FirstGroupMemberFirstName & \FirstGroupMemberMatricleNumber \\
\textbf{2}    & \SecondGroupMemberLastName & \SecondGroupMemberFirstName & \SecondGroupMemberMatricleNumber \\
\textbf{3}    & \ThirdGroupMemberLastName & \ThirdGroupMemberFirstName & \ThirdGroupMemberMatricleNumber \\
\textbf{4}    & \FourthGroupMemberLastName & \FourthGroupMemberFirstName & \FourthGroupMemberMatricleNumber \\
\bottomrule
\end{tabular}
\end{table}

\flushleft

\par \textbf{Benötigte Dateien}\\
Alle benötigten Datensätze und Skriptvorlagen finden Sie in unserem Moodle-Kurs: \urlc{https://moodle.informatik.tu-darmstadt.de/course/view.php?id=937}

\textbf{Gruppeneinteilung}\\
Bearbeiten Sie diese Übung in Dreier- oder Vierergruppen. Es steht Ihnen frei die Gruppen selbst zu bilden. Nutzen Sie hierfür die Gruppeneinteilung und das Forum in Moodle.
Sehen Sie bitte aufgrund der aktuellen Coronakrise davon ab, sich lokal zu treffen und nutzen Sie stattdessen digitale Kommunikationskanäle.
Zur gemeinsamen Bearbeitung der Abgabe können sie beispielsweise \urlc{https://overleaf.com}~nutzen.

\textbf{Theoretische Aufgaben}\\
Bei theoretischen Übungsaufgaben, sind wir Ihnen sehr dankbar, wenn Sie diese in \LaTeX~formatieren und als PDF einreichen.
Nutzen Sie hierfür die \LaTeX-Vorlage und die vorgesehene Blöcke:
\begin{verbatim}
\begin{solution}
% Geben sie hier ihre Antwort an.
\end{solution}
\end{verbatim}

Geben Sie dabei ihre Gruppenmitglieder und Gruppennummber in der Datei \texttt{group\_members.tex} an.

Wenn Sie mit \LaTeX~nicht ausreichend vertraut sind, können Sie auch einen hochauflösenden Scan einer handgeschriebenen Lösung einreichen.
Bitte schreiben Sie ordentlich und leserlich.

\textbf{Programmieraufgaben}\\
Bei Aufgaben, die mit einem \codesym~versehen sind, handelt es sich um Programmieraufgaben.
Bearbeiten Sie in diesem Fall die vorgegebene Programmiervorlage.
Verwenden Sie bevorzugt \textbf{Python 3.7}, da wir diese Version zum Testen ihrer Lösung benutzen.
Benennen Sie die Funktionsdateien nicht um und ändern Sie die angegebenen Funktionssignaturen nicht. Wenn Sie das Gefühl haben, dass es ein
Fehler bei den Zuweisungen, fragen Sie uns auf Moodle.

\textbf{Formalien zur Abgabe}\\
Bitte laden Sie Ihre Lösungen in der entsprechenden Rubrik auf Moodle hoch. Sie müssen \textbf{nur eine Lösung pro Gruppe} einreichen. Wenn Sie keinen Zugang zu Moodle haben, setzen Sie sich bitte so schnell wie möglich mit uns in Verbindung. Laden Sie alle Ihre Lösungsdateien (die PDF-Abgabe und .py-Dateien) als eine einzige .zip-Datei hoch. Bitte beachten Sie, dass wir keine anderen als die angegebenen Dateiformate akzeptieren.
\textbf{Laden Sie den gegeben Datensatz zur Programmieraufgabe nich in der Abgabe hoch.
}
Nutzen Sie folgende Namensgebung:

\vspace{0.2cm}
\dirtree{%
.1 {dmml\_bonus1\_group<groupid>.zip}.
.2 dmml\_bonus1.pdf.
.2 02\_dt\_classification.py.
.2 03\_dt\_regression.py.
.2 04\_random\_forest.py.
}

\textbf{Verspätete Abgaben}\\
Verspätete Abgaben werden akzeptiert, aber für jeden Tag, an dem die Abgabefirst überschritten wird, werden 25\,\% der insgesamt erreichbaren Punkte abgezogen. Nachdem die Übung offiziell besprochen wurde, können Sie die Aufgabe nicht mehr einreichen.

\textbf{Bewertungsfaktoren}\\
Die Bewertung dieser Übung hängt von den folgenden Faktoren ab:
\begin{itemize}
    \item Richtigkeit der Antwort
    \item Klarheit der Präsentation der Ergebnisse
    \item Schreibstil
\end{itemize}
Wenn Sie bei einer Aufgabe nicht weiterkommen, versuchen Sie zu erklären warum und beschreiben Sie die Probleme, auf die Sie gestoßen sind,
da Sie dafür Teilpunkte erhalten können.


\textbf{Umgang mit Plagiaten}\\
Sie dürfen gerne kursbezogene Themen in der Vorlesung oder in unseren Moodle Foren diskutieren.
Sie sollten allerdings keine Lösungen mit anderen Gruppen teilen, und alles, was Sie einreichen, muss Ihre eigene Arbeit sein. Es ist Ihnen auch nicht gestattet, Material aus dem Internet zu kopieren. Sie sind verpflichtet, jede Informationsquelle, die Sie zur Lösung der Übungsaufgabe verwendet haben (d.h. andere Materialien als die Vorlesungsmaterialien), anzuerkennen. Zitierungen haben keinen Einfluss auf Ihre Note. Nicht anerkennen einer Quelle, die Sie verwendet haben, ist dagegen ein klarer Verstoß gegen die akademische Ethik. Beachten Sie, dass die Universität sehr ernst mit Plagiaten umgeht.
